

% [[last here]]

[[concrete method]]

The first step of any basis-dependent many-body method is to compute the matrix elements of the one-body and two-body Hamiltonian operators.  In a harmonic oscillator basis, the one-body Hamiltonian is diagonal.  Its diagonal matrix elements are computed directly using \label{eq:energysingleparticlestate}.

The more difficult part is the computation of two-body matrix elements, which requires first the calculating the integration integrals through \label{eq:interactionintegral} and then antisymmetrizing them with \label{eq:antisymmetricmatrixelement}.  One possibility is to use a analytic formula involving iterated summations and products as described in \cite{0953-8984-10-3-013}.  However, the formula has a rapidly growing asymptotic computational cost and also becomes increasingly sensitive to numerical losses due to oscillations as the shell number increases.  Therefore, we opted to use an alternative approach where integrals are calculated exactly in the center-of-mass frame using numerical quadrature and then transformed into the laboratory frame.  This part was done using OpenFCI software, which describes the technique in detail \cite{2008arXiv0810.2644K}.

With the Hamiltonian

The use of second quantization enables the application of an extremely useful combinatoric theorem known as Wick's theorem \cite{shavitt2009many}, which provides systematic way to expand complicated expressions involving creation and annihilation operators.  Ultimately, after simplification, computations \textit{in silico} do not involve algebraic manipulation of abstract operators at all -- rather, they involve numerical calculations of the \emph{matrix elements} via equations derived using Wick's theorem.
